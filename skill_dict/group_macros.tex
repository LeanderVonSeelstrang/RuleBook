\usepackage{hyperref}
\usepackage{xparse}
\usepackage{etoolbox}
\usepackage{tabularx}

\usepackage{environ,needspace}
\usepackage{enumitem} % To use [leftmargin=*]

\usepackage{xcolor}
\definecolor{myturquoise}{RGB}{25, 170, 160} % Define turquoise-blue color to highlight some groups

\newsavebox{\groupbox} % New savebox for the Group environment to determine if new page is needed

\newcounter{groupcounter}

% Use the environment to define a Group - this is just a helper
\newcommand{\DefGroup}[4]{%
  \stepcounter{groupcounter}%
  \edef\temp{\arabic{groupcounter}}% Capture the current value of the counter
  \expandafter\xdef\csname groupnum#1\endcsname{\temp}%
  \csgdef{groupid#1}{#1}
  \csgdef{groupdesc#1}{#2}
  \csgdef{groupbonus#1}{#3}
  \csgdef{grouptype#1}{#4}
}

% The Group environment is used to define a group of clustered skills. The teams are supposed to be successful with every subtask, to unlock the bonus. 
% If a subgroup is a Selection Group you only need to be successful once to be able to unlock the bonus of the super group. 

% Example usage:
% Handle Obstacles (Selection Group)
%\begin{Group}{HandleObstacles}{Handle obstacles such as avoiding objects, marked areas, and people during navigation}{40}{Selection}
%    \Skill{AvoidNavigationObstacle} % AvoidNavigationObstacle is the SkillId (see skill dictionary)
%\end{Group}
%\begin{Group}{FollowSection}{Follow a person while handling obstacles}{20}{Sequential}
%    \Skill{TrackFollowPerson}  % Reference to a skill
%    \GroupRef{HandleObstacles} % Reference to another group
%\end{Group}

% \begin{Group}{GroupId}{Description}{Bonus Points}{Group Type}
\NewDocumentEnvironment{Group}{mmmm}{%
  \DefGroup{#1}{#2}{#3}{#4}%
   \par\noindent
   \begin{lrbox}{\groupbox} % Start the lrbox 
  \begin{minipage}{\linewidth}
    %\noindent\textbf{Group} (G-\csuse{groupnum#1}) \hfill \textbf{Type:} #4 \quad \textbf{Bonus:} #3\\
    \begin{tabularx}{\linewidth}{@{}lXl@{\hspace{7em}}l@{}} 
      \textbf{Group} (G-\csuse{groupnum#1}) & & \makebox[5em][l]{\textbf{Type:} #4} & \makebox[3em][r]{\textbf{Bonus:} #3} \\
    \end{tabularx}
    \quad #2    %Description: #2
    \label{grp:G-#1} % Create a label for the group
    \vspace{0.31em} % Add space before the group
    \hrule % Horizontal line before the group
    \begin{itemize}[leftmargin=*, itemsep=-0.6pt, topsep=1.0pt]
    \renewcommand{\labelitemi}{} % Hide bullet points
}{
    \end{itemize}
  \end{minipage} 
   \end{lrbox}% End the lrbox
     \Needspace{\dimexpr\ht\groupbox+\dp\groupbox+.5\baselineskip}% Check space requirement
   \usebox{\groupbox}% Output the content
   \par\addvspace{0.4\baselineskip}% Add consistent space after the Group
   \aftergroup\aftergroup\aftergroup\noindent 
}

% \GroupRef{GroupId}[Optional New Description]
\NewDocumentCommand{\GroupRef}{m o}{%
  \ifcsdef{groupid#1}{%
      \item \begin{tabularx}{\linewidth}{@{}p{1cm}X@{}}
        {\hyperref[grp:G-#1]{\textbf{G-\csuse{groupnum#1}}}} &
        \textit{\csuse{groupdesc#1}}
        \IfValueT{#2}{#2} % Remark
      \end{tabularx}
    }{%
      \item \textbf{\textcolor{red}{Group with ID '#1' not found.}}
    }
}

