\section{Hand Me That [Party Host]}
\label{test:hand-me-that}
A guest at the party speaks English, but with only a limited vocabulary. The robot will assist them in obtaining things that they gesture for.\\
% Comments: In this tasks each group of items is scored separately. Hence 

%\subsection{Focus}
%Joint attention is a well-studied and important task in Human-Robot Interaction. The goal of this task is to really challenge the teams to perform a hard HRI task.


\noindent \textbf{Main Goal:} The robot identifies (touching or naming) each object at which the operator is pointing at.

\subsection*{Focus}
\emph{Object perception}, \emph{HRI}


% %% %%%%%%%%%%%%%%%%%%%%%%%%%%%%%%%%%%%%%%%%%%%%%%%%%%%%%%
%
% Setup
%
% %% %%%%%%%%%%%%%%%%%%%%%%%%%%%%%%%%%%%%%%%%%%%%%%%%%%%%%%
\subsection*{Setup}
\begin{enumerate}[nosep]
	\item \textbf{Location:} 
	\begin{itemize}[nosep]
		\item This takes place in a room in the \Arena{}.
		\item The robot and the operator stand in a predefined starting location announced beforehand % (OC instructions: announce this 2 hours before the test).
	\end{itemize}
	    \item \textbf{Objects:} 
		\begin{itemize}[nosep] 
		\item \textbf{Groups of Objects: }There are five groups of 2--5 objects randomly placed along the room.
		\item \textbf{Deck:} The referee has a deck of objects to request, one per group, sorted by distance.
		\end{itemize}

\end{enumerate}


% %% %%%%%%%%%%%%%%%%%%%%%%%%%%%%%%%%%%%%%%%%%%%%%%%%%%%%%%
%
% Procedure
%
% %% %%%%%%%%%%%%%%%%%%%%%%%%%%%%%%%%%%%%%%%%%%%%%%%%%%%%%%
\subsection*{Procedure}
\begin{enumerate}[nosep]
	\item For each group of objects, the robot asks the operator: \emph{What do you need?}.  % We rule out natural language interaction
		\item The operator walks near to the object and points at it.
		\item The robot asks as many questions as necessary. 
		\item The operator replies to each question (most likely with \emph{yes}, \emph{no}, \emph{I don't know}, single word, etc).
		\\\textbf{Remark:} The operator does not know the name of the object.
\end{enumerate}


% %% %%%%%%%%%%%%%%%%%%%%%%%%%%%%%%%%%%%%%%%%%%%%%%%%%%%%%%
%
% Additional Rules
%
% %% %%%%%%%%%%%%%%%%%%%%%%%%%%%%%%%%%%%%%%%%%%%%%%%%%%%%%%
\subsection*{Additional rules and remarks}
\begin{enumerate}[nosep]
	\item \textbf{Keep going:} The robot should keep trying to determine the referred object until they score or run out of time.

	\item \textbf{Skipping groups:} The robot may say \emph{Pass} or \emph{I give up} to try with the next object.

	\item \textbf{Incorrect guesses:} Incorrect guesses reduce the value of the correct guess the first two times. Guessing correctly on the third or fourth attempt is worth 100 points. After the fourth guess is worth no points.

	\item\textbf{Colors and categories:} Asking for the color or category of a pointed object applies a penalty of 400 points for that particular object.

	\item\textbf{Uneducated operator:} The referee may instruct the operator to answer \emph{I don't understand} or \emph{I don't know} if the robot asks complex questions or is attempting blind guessing.

	\item \textbf{Groups of Objects:} A group consists of 2--5 random standard objects (see~\refsec{rule:scenario_objects}), separated one from another. They can be in a close proximity but not touching.
	The average distance between the starting position and each group ranges between 50cm and 150cm.

\end{enumerate}

\subsection*{Instructions:}

\subsubsection*{To Referee}

The referee needs to:
\begin{itemize}[nosep]
	\item Rearrange and mix groups between runs.
	\item Verify that the operator is pointing at the right item.
\end{itemize}

\subsubsection*{To OC}
The OC needs to:
\begin{itemize}[nosep]
	\item \textbf{On setup day}: Announce the starting position of the robot.
\end{itemize}



\subsection*{Score sheet}
\section{Hand Me That [Party Host]}
\label{test:hand-me-that}
A guest at the party speaks English, but with only a limited vocabulary. The robot will assist them in obtaining things that they gesture for.\\
% Comments: In this tasks each group of items is scored separately. Hence 

%\subsection{Focus}
%Joint attention is a well-studied and important task in Human-Robot Interaction. The goal of this task is to really challenge the teams to perform a hard HRI task.


\noindent \textbf{Main Goal:} The robot identifies (touching or naming) each object at which the operator is pointing at.

\subsection*{Focus}
\emph{Object perception}, \emph{HRI}


% %% %%%%%%%%%%%%%%%%%%%%%%%%%%%%%%%%%%%%%%%%%%%%%%%%%%%%%%
%
% Setup
%
% %% %%%%%%%%%%%%%%%%%%%%%%%%%%%%%%%%%%%%%%%%%%%%%%%%%%%%%%
\subsection*{Setup}
\begin{enumerate}[nosep]
	\item \textbf{Location:} 
	\begin{itemize}[nosep]
		\item This takes place in a room in the \Arena{}.
		\item The robot and the operator stand in a predefined starting location announced beforehand % (OC instructions: announce this 2 hours before the test).
	\end{itemize}
	    \item \textbf{Objects:} 
		\begin{itemize}[nosep] 
		\item \textbf{Groups of Objects: }There are five groups of 2--5 objects randomly placed along the room.
		\item \textbf{Deck:} The referee has a deck of objects to request, one per group, sorted by distance.
		\end{itemize}

\end{enumerate}


% %% %%%%%%%%%%%%%%%%%%%%%%%%%%%%%%%%%%%%%%%%%%%%%%%%%%%%%%
%
% Procedure
%
% %% %%%%%%%%%%%%%%%%%%%%%%%%%%%%%%%%%%%%%%%%%%%%%%%%%%%%%%
\subsection*{Procedure}
\begin{enumerate}[nosep]
	\item For each group of objects, the robot asks the operator: \emph{What do you need?}.  % We rule out natural language interaction
		\item The operator walks near to the object and points at it.
		\item The robot asks as many questions as necessary. 
		\item The operator replies to each question (most likely with \emph{yes}, \emph{no}, \emph{I don't know}, single word, etc).
		\\\textbf{Remark:} The operator does not know the name of the object.
\end{enumerate}


% %% %%%%%%%%%%%%%%%%%%%%%%%%%%%%%%%%%%%%%%%%%%%%%%%%%%%%%%
%
% Additional Rules
%
% %% %%%%%%%%%%%%%%%%%%%%%%%%%%%%%%%%%%%%%%%%%%%%%%%%%%%%%%
\subsection*{Additional rules and remarks}
\begin{enumerate}[nosep]
	\item \textbf{Keep going:} The robot should keep trying to determine the referred object until they score or run out of time.

	\item \textbf{Skipping groups:} The robot may say \emph{Pass} or \emph{I give up} to try with the next object.

	\item \textbf{Incorrect guesses:} Incorrect guesses reduce the value of the correct guess the first two times. Guessing correctly on the third or fourth attempt is worth 100 points. After the fourth guess is worth no points.

	\item\textbf{Colors and categories:} Asking for the color or category of a pointed object applies a penalty of 400 points for that particular object.

	\item\textbf{Uneducated operator:} The referee may instruct the operator to answer \emph{I don't understand} or \emph{I don't know} if the robot asks complex questions or is attempting blind guessing.

	\item \textbf{Groups of Objects:} A group consists of 2--5 random standard objects (see~\refsec{rule:scenario_objects}), separated one from another. They can be in a close proximity but not touching.
	The average distance between the starting position and each group ranges between 50cm and 150cm.

\end{enumerate}

\subsection*{Instructions:}

\subsubsection*{To Referee}

The referee needs to:
\begin{itemize}[nosep]
	\item Rearrange and mix groups between runs.
	\item Verify that the operator is pointing at the right item.
\end{itemize}

\subsubsection*{To OC}
The OC needs to:
\begin{itemize}[nosep]
	\item \textbf{On setup day}: Announce the starting position of the robot.
\end{itemize}



\subsection*{Score sheet}
\section{Hand Me That [Party Host]}
\label{test:hand-me-that}
A guest at the party speaks English, but with only a limited vocabulary. The robot will assist them in obtaining things that they gesture for.\\
% Comments: In this tasks each group of items is scored separately. Hence 

%\subsection{Focus}
%Joint attention is a well-studied and important task in Human-Robot Interaction. The goal of this task is to really challenge the teams to perform a hard HRI task.


\noindent \textbf{Main Goal:} The robot identifies (touching or naming) each object at which the operator is pointing at.

\subsection*{Focus}
\emph{Object perception}, \emph{HRI}


% %% %%%%%%%%%%%%%%%%%%%%%%%%%%%%%%%%%%%%%%%%%%%%%%%%%%%%%%
%
% Setup
%
% %% %%%%%%%%%%%%%%%%%%%%%%%%%%%%%%%%%%%%%%%%%%%%%%%%%%%%%%
\subsection*{Setup}
\begin{enumerate}[nosep]
	\item \textbf{Location:} 
	\begin{itemize}[nosep]
		\item This takes place in a room in the \Arena{}.
		\item The robot and the operator stand in a predefined starting location announced beforehand % (OC instructions: announce this 2 hours before the test).
	\end{itemize}
	    \item \textbf{Objects:} 
		\begin{itemize}[nosep] 
		\item \textbf{Groups of Objects: }There are five groups of 2--5 objects randomly placed along the room.
		\item \textbf{Deck:} The referee has a deck of objects to request, one per group, sorted by distance.
		\end{itemize}

\end{enumerate}


% %% %%%%%%%%%%%%%%%%%%%%%%%%%%%%%%%%%%%%%%%%%%%%%%%%%%%%%%
%
% Procedure
%
% %% %%%%%%%%%%%%%%%%%%%%%%%%%%%%%%%%%%%%%%%%%%%%%%%%%%%%%%
\subsection*{Procedure}
\begin{enumerate}[nosep]
	\item For each group of objects, the robot asks the operator: \emph{What do you need?}.  % We rule out natural language interaction
		\item The operator walks near to the object and points at it.
		\item The robot asks as many questions as necessary. 
		\item The operator replies to each question (most likely with \emph{yes}, \emph{no}, \emph{I don't know}, single word, etc).
		\\\textbf{Remark:} The operator does not know the name of the object.
\end{enumerate}


% %% %%%%%%%%%%%%%%%%%%%%%%%%%%%%%%%%%%%%%%%%%%%%%%%%%%%%%%
%
% Additional Rules
%
% %% %%%%%%%%%%%%%%%%%%%%%%%%%%%%%%%%%%%%%%%%%%%%%%%%%%%%%%
\subsection*{Additional rules and remarks}
\begin{enumerate}[nosep]
	\item \textbf{Keep going:} The robot should keep trying to determine the referred object until they score or run out of time.

	\item \textbf{Skipping groups:} The robot may say \emph{Pass} or \emph{I give up} to try with the next object.

	\item \textbf{Incorrect guesses:} Incorrect guesses reduce the value of the correct guess the first two times. Guessing correctly on the third or fourth attempt is worth 100 points. After the fourth guess is worth no points.

	\item\textbf{Colors and categories:} Asking for the color or category of a pointed object applies a penalty of 400 points for that particular object.

	\item\textbf{Uneducated operator:} The referee may instruct the operator to answer \emph{I don't understand} or \emph{I don't know} if the robot asks complex questions or is attempting blind guessing.

	\item \textbf{Groups of Objects:} A group consists of 2--5 random standard objects (see~\refsec{rule:scenario_objects}), separated one from another. They can be in a close proximity but not touching.
	The average distance between the starting position and each group ranges between 50cm and 150cm.

\end{enumerate}

\subsection*{Instructions:}

\subsubsection*{To Referee}

The referee needs to:
\begin{itemize}[nosep]
	\item Rearrange and mix groups between runs.
	\item Verify that the operator is pointing at the right item.
\end{itemize}

\subsubsection*{To OC}
The OC needs to:
\begin{itemize}[nosep]
	\item \textbf{On setup day}: Announce the starting position of the robot.
\end{itemize}



\subsection*{Score sheet}
\section{Hand Me That [Party Host]}
\label{test:hand-me-that}
A guest at the party speaks English, but with only a limited vocabulary. The robot will assist them in obtaining things that they gesture for.\\
% Comments: In this tasks each group of items is scored separately. Hence 

%\subsection{Focus}
%Joint attention is a well-studied and important task in Human-Robot Interaction. The goal of this task is to really challenge the teams to perform a hard HRI task.


\noindent \textbf{Main Goal:} The robot identifies (touching or naming) each object at which the operator is pointing at.

\subsection*{Focus}
\emph{Object perception}, \emph{HRI}


% %% %%%%%%%%%%%%%%%%%%%%%%%%%%%%%%%%%%%%%%%%%%%%%%%%%%%%%%
%
% Setup
%
% %% %%%%%%%%%%%%%%%%%%%%%%%%%%%%%%%%%%%%%%%%%%%%%%%%%%%%%%
\subsection*{Setup}
\begin{enumerate}[nosep]
	\item \textbf{Location:} 
	\begin{itemize}[nosep]
		\item This takes place in a room in the \Arena{}.
		\item The robot and the operator stand in a predefined starting location announced beforehand % (OC instructions: announce this 2 hours before the test).
	\end{itemize}
	    \item \textbf{Objects:} 
		\begin{itemize}[nosep] 
		\item \textbf{Groups of Objects: }There are five groups of 2--5 objects randomly placed along the room.
		\item \textbf{Deck:} The referee has a deck of objects to request, one per group, sorted by distance.
		\end{itemize}

\end{enumerate}


% %% %%%%%%%%%%%%%%%%%%%%%%%%%%%%%%%%%%%%%%%%%%%%%%%%%%%%%%
%
% Procedure
%
% %% %%%%%%%%%%%%%%%%%%%%%%%%%%%%%%%%%%%%%%%%%%%%%%%%%%%%%%
\subsection*{Procedure}
\begin{enumerate}[nosep]
	\item For each group of objects, the robot asks the operator: \emph{What do you need?}.  % We rule out natural language interaction
		\item The operator walks near to the object and points at it.
		\item The robot asks as many questions as necessary. 
		\item The operator replies to each question (most likely with \emph{yes}, \emph{no}, \emph{I don't know}, single word, etc).
		\\\textbf{Remark:} The operator does not know the name of the object.
\end{enumerate}


% %% %%%%%%%%%%%%%%%%%%%%%%%%%%%%%%%%%%%%%%%%%%%%%%%%%%%%%%
%
% Additional Rules
%
% %% %%%%%%%%%%%%%%%%%%%%%%%%%%%%%%%%%%%%%%%%%%%%%%%%%%%%%%
\subsection*{Additional rules and remarks}
\begin{enumerate}[nosep]
	\item \textbf{Keep going:} The robot should keep trying to determine the referred object until they score or run out of time.

	\item \textbf{Skipping groups:} The robot may say \emph{Pass} or \emph{I give up} to try with the next object.

	\item \textbf{Incorrect guesses:} Incorrect guesses reduce the value of the correct guess the first two times. Guessing correctly on the third or fourth attempt is worth 100 points. After the fourth guess is worth no points.

	\item\textbf{Colors and categories:} Asking for the color or category of a pointed object applies a penalty of 400 points for that particular object.

	\item\textbf{Uneducated operator:} The referee may instruct the operator to answer \emph{I don't understand} or \emph{I don't know} if the robot asks complex questions or is attempting blind guessing.

	\item \textbf{Groups of Objects:} A group consists of 2--5 random standard objects (see~\refsec{rule:scenario_objects}), separated one from another. They can be in a close proximity but not touching.
	The average distance between the starting position and each group ranges between 50cm and 150cm.

\end{enumerate}

\subsection*{Instructions:}

\subsubsection*{To Referee}

The referee needs to:
\begin{itemize}[nosep]
	\item Rearrange and mix groups between runs.
	\item Verify that the operator is pointing at the right item.
\end{itemize}

\subsubsection*{To OC}
The OC needs to:
\begin{itemize}[nosep]
	\item \textbf{On setup day}: Announce the starting position of the robot.
\end{itemize}



\subsection*{Score sheet}
\input{scoresheets/HandMeThat.tex}

% Local Variables:
% TeX-master: "Rulebook"
% End:


% Local Variables:
% TeX-master: "Rulebook"
% End:


% Local Variables:
% TeX-master: "Rulebook"
% End:


% Local Variables:
% TeX-master: "Rulebook"
% End:
